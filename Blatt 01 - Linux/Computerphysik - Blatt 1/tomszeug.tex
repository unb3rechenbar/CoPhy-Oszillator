\usepackage{calrsfs}
\usepackage{thmtools}
\usepackage[usenames,dvipsnames]{xcolor}
\declaretheoremstyle[
    			headfont=\bfseries\sffamily\color{black!70!black}, bodyfont=\normalfont,
    			mdframed={
       				linewidth=1pt,
        			rightline=false, topline=false, bottomline=false,
    			}
			]{Begruendungsbox}
\declaretheoremstyle[
    			headfont=\bfseries\sffamily\color{black!70!black}, bodyfont=\normalfont,
    			mdframed={
       				linewidth=1pt,
        			rightline=false, topline=false, bottomline=false,
    			}
			]{Erinnerungsbox}
\declaretheoremstyle[
    			headfont=\bfseries\sffamily\color{black!70!black}, bodyfont=\normalfont, numbered=no,
    			mdframed={
       				linewidth=false,
        			rightline=false, topline=false, bottomline=false,
    			}
			]{Zusammenfassungsbox}
\declaretheoremstyle[
    			headfont=\bfseries\sffamily\color{black!70!black}, bodyfont=\normalfont, numbered=no,
    			mdframed={
       				linewidth=0.5pt,
        			rightline=0.5pt, topline=false, bottomline=false,
    			}
			]{Voraussetzungsbox}
\declaretheorem[style=Voraussetzungsbox, name=Voraussetzung]{vrr}
\declaretheorem[style=Zusammenfassungsbox, name=Zusammenfassen]{zsmfs}
\newenvironment{Voraussetzung}{\begin{vrr}}{\end{vrr}}
\newenvironment{Zusammenfassen}{\begin{zsmfs}}{\end{zsmfs}}
\declaretheorem[style=Erinnerungsbox, name=Vermutung]{verm}
\declaretheorem[style=Begruendungsbox, name=Begründung]{begr}
\newenvironment{begruendung}{\begin{begr}}{\end{begr}}

\newcommand{\nbra}[1]{\left(#1\right)}
\newcommand{\nsqbra}[1]{\left[#1\right]}
\newcommand{\nset}[1]{\left\{{#1}\right\}}
\newcommand{\lfdef}[2]{\begin{pmatrix}{#1}\\{#2}\end{pmatrix}}
\newcommand{\fdef}[2]{\nbra{#1}_{#2}}
\newcommand{\dabs}[2]{\abs{\abs{#1}}_{#2}}
\newcommand{\Abb}[2]{\text{Abb}\nbra{#1,#2}}
\newcommand{\postref}[1]{\nsqbra{\rightarrow{#1}}}
\newcommand{\anm}[1]{[\textit{#1}]}

\declaretheorem[style=Erinnerungsbox, name=Erinnerung]{Erinnerungsdef}
\newenvironment{Erinnerung}{\begin{Erinnerungsdef}}{\end{Erinnerungsdef}}

\renewcommand{\vec}[1]{\mathbf{#1}}

\newcommand{\diff}[4]{%
			\ifthenelse{\isempty{#2}}
				{\ifthenelse{\isempty{#3}}
					{\ifthenelse{\isempty{#4}}
						{\ensuremath{d#1}}
						{\ensuremath{d^{#4}#1}}}
					{\ifthenelse{\isempty{#4}}
						{\ensuremath{d#1\nbra{#3}}}
						{\ensuremath{d^{#4}#1\nbra{#3}}}}}
				{\ifthenelse{\isempty{#3}}
					{\ifthenelse{\isempty{#4}}
						{\ensuremath{d#1()\nbra{#2}}}
						{\ensuremath{d^{#4}#1()\nbra{#2}}}}
					{\ifthenelse{\isempty{#4}}
						{\ensuremath{d#1\nbra{#3}\nbra{#2}}}
						{\ensuremath{d^{#4}#1\nbra{#3}\nbra{#2}}}}}%
		}
\newcommand{\Diff}[4]{%
			\ifthenelse{\isempty{#3}}
				{\ifthenelse{\isempty{#4}}
					{\ensuremath{\text{D}_{#2}#1}}
					{\ensuremath{\text{D}_{#2}^{#4}#1}}}
				{\ifthenelse{\isempty{#4}}
					{\ensuremath{\text{D}_{#2}#1\nbra{#3}}}
					{\ensuremath{\text{D}_{#2}^{#4}#1\nbra{#3}}}}%
		}
\newcommand{\pdiff}[4]{%
			\ifthenelse{\isempty{#3}}
				{\ifthenelse{\isempty{#4}}
					{\ensuremath{\partial_{#2}#1}}
					{\ensuremath{\partial_{#2}^{#4}#1}}}
				{\ifthenelse{\isempty{#4}}
					{\ensuremath{\partial_{#2}#1\nbra{#3}}}
					{\ensuremath{\partial_{#2}^{#4}#1\nbra{#3}}}}%
		}
\newcommand{\scpr}[2]{\left\langle #1,#2\right\rangle}
\DeclareMathOperator{\mal}{mal}
\DeclareMathOperator{\plus}{plus}
\renewcommand{\neg}{\text{neg}}
\DeclareMathOperator{\inv}{inv}

\newcommand\numberthis{\addtocounter{equation}{1}\tag{\theequation}}
\newenvironment{beh}{\noindent\textbf{\textit{Behauptung.}}}{}
\DeclareMathOperator{\laplace}{\text{lap}}
\newcommand{\Null}[1]{0_{#1}}

\newcommand{\gint}[1]{\int_{[#1]}}
\newcommand{\dint}[2][]{%
				\ifthenelse{\isempty{#2}}{\,d#2}{\,d_{#1}#2}
				}%
\newcommand{\Ell}{\mathcal L}
\newcommand{\Real}[1]{\text{Re}\nbra{#1}}
\newcommand{\Imag}[1]{\text{Im}\nbra{#1}}

\newcommand{\Mengeninklusion}[2]{
	\begin{itemize}
		\item[\enquote{$\subseteq$}] #1
		\item[\enquote{$\supseteq$}] #2
	\end{itemize}}
 \newcommand{\sAlgebra}[1]{{\sigma}\Mengenschriftdesign{Algebra}\nbra{#1}}

\usepackage{xargs}
\newcounter{ex}
			\newcounter{subex}[ex]
			\newcounter{subsubex}[subex]
			
			\setcounter{ex}{1}
 \newcommandx{\aufgabe}[1][1={}]{\ifnumgreater{\theex}{1}{\newpage}{}
				\ifthenelse{\isempty{#1}}	
					{\section*{Aufgabe \theex}\label{ex:\theex}\stepcounter{ex}}
					{\section*{Aufgabe #1}\label{ex:#1}}}
\newcommandx{\subaufgabe}[1][1={}]{\ifthenelse{\isempty{#1}}
					{\stepcounter{subex}\subsection*{(\alph{subex})}\label{subex:\theex-\thesubex}}
					{\stepcounter{subex}\subsection*{(#1)}\label{subex:\theex-#1}}}