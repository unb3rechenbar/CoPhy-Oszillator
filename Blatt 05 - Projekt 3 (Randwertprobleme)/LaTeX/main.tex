\documentclass[
    oneside,
    ngerman,
    footinclude=false,
    captions=tableheading,
    DIV=12
]{scrartcl}


\usepackage{UniLaTeXPackage}

\ihead{Tom Folgmann,\\David Jannack}
\chead{Daniel Kazenwadel\\Blatt 05}
\ohead{\today}

\begin{document}
    \aufgabe{}
        \subaufgabe{}
            Zur Vorbereitung der Implementierung betrachten wir die gegebene \emph{Poisson Differentialgleichung} der Form $d^2\Phi = \mcf$ für $\Phi\in C^2(\R^3,\R)$ und $\mcf\in C^0(\R^3,\R)$. Für die gegebene Ladungsverteilung $\rho:=\big(\exp(-r)/(8\cdot\pi)\big)_{r\in\R_{>0}}$ können wir die DGL unter Drehsymmetrie reduzieren auf $d^2\varphi = f$, für angepasstes $\varphi\in C^2(\R)$ und $f\in C^0(\R)$. Mit der Definition $f := \big(-r\cdot \rho(r) / \epsilon\big)_{(r,x)\in\R\times\R}$ für $\epsilon\in\R^\times$ und können wir dann eine rechte Seite $F$ definieren durch 
            \[
                d^2(u)(r)(1) = u''(r) = \ubra{(f\circ (\id_\R,u))(r)}{F(r,u(r))}.
            \] 
            Daraus ergibt sich also die nicht autonome rechte Seite $F(t,x) = -\rho(t)\cdot t / \epsilon$. Da die Numerov Methode auf DGL der Form $d^2 u + (k^2\cdot u) = f$ angewendet wird, müssen können wir $k = 0$ voraussetzen und haben damit die passende Form: 
            \[
                d^2(u)(r)(1) = u''(r) = F(r,u(r)).
            \]
            Durch einen glücklichen Umstand fallen die Definitionen des $F_\textit{Vl}$ aus der Vorlesung mit unserem $F$ durch den Zusammenhang $F_\textit{Vl}: r\mapsto F(r,u(r))$ Dadurch fallen entscheidende Terme in der Berechnung des Folgegliedes von $F_\textit{Vl}(r) = F(r,u(r))$ weg und wir finden 
            \[
                u(t_{i+1}) = \frac{h^2\cdot (F(r_{i+1},u(r_{i+1})) + 10\cdot F(r_i,u(r_i)) + F(r_{i-1},u(r_{i-1})))/12 + 2\cdot u(r_i) - u(r_{i-1})}{1} + \mcO(h^6).
            \]
            Wir erkennen also, daß wir in der Programmierung das Tupel $(u(r_{i}),u(r_{i-1}))$ zu jeder Berechnung speichern müssen. 

            \subsubsection*{Analytische Lösung}
                Die analytische Lösung des beschriebenen DGPs ist gegeben durch 
                \[
                    v := \nbra{-\frac{\exp(-r)}{4\cdot\pi\cdot\epsilon}\cdot\nbra{1 + \frac{r}{2}} + C_1\cdot r + C_2}_{r\in\R_{>0}}.
                \]
                Mit den Randbedingungen $\lim_{r\to\infty}v(r) = 0$ und $\lim_{r\to 0} v(r) = 1/(4\pi\epsilon)$ folgern wir für die Konstanten $C_1$ und $C_2$ die Werte $C_1 = 0$ und $C_2 = 1/(4\pi\epsilon)$. Damit ergibt sich die analytische Lösung zu
                \[
                    v(r) = -\frac{\exp(-r)}{4\cdot\pi\cdot\epsilon}\cdot\nbra{1 + \frac{r}{2}} + \frac{1}{4\pi\epsilon}.
                \]
        \subaufgabe{}

    \aufgabe{}
        \subaufgabe{}
            Wir stellen zunächst die Schrödingergleichung $Hf = \lambda\cdot f$ für $f\in C^2(\R)$ in die nötige Form $d^2f = \fdef{F(t,f(t))}_{t\in D_f}$ um. Wir finden 
            \begin{align*}
                \nbra{-\frac{\hbar^2}{2m}\cdot d^2}(f) + f\cdot V = \lambda\cdot f\iff d^2f = \frac{2m}{\hbar^2}\cdot\big(\lambda - V\big)\cdot f.
            \end{align*}
            Wir definieren also die durch $\N$ abgezählte rechte Seite als
            \[
                n\mapsto F_n:=\fdef{\frac{2m}{\hbar^2}\cdot\big(\lambda_n - V(x)\big)\cdot f_n(x)}{t\in D_f},\quad V:=\fdef{\begin{cases}
                    x & x\geq 0\\
                    \infty & \text{sonst}
                \end{cases}}{x\in D_f}.
            \]
            Wir setzen $D_f:=\R$ und implementieren die rechte Seite.

    \aufgabe{}
        Als Lösung einer gewönlichen Differentialgleichung erster Ordnung können wir durch Integration eine Funktion definieren, welche uns zu beliebigem Startwert $S\in\R^2$ eine Lösung $u$ liefert. Diese Funktion ist gegeben durch
        \begin{align*}
            A^{-1}_{t_0,F}:=\bigl(V(t_0,S,F)\bigr)_{S\in\R^2},
        \end{align*}
        wobei wir $V$ als Verfahrensfunktion zur Lösung eines AWPs der Form $u''(t) + k^2\cdot u(t) = f(t)$, $u(t_0) = S_1$ und $u'(t_0) = S_2$ annehmen. Hier wird später das Numerov Verfahren verwendet werden. Die Ausgabe dieser Funktion $A^{-1}_{t_0,F}(S) $ nehmen wir nun als Eingabe einer zweiten Funktion $B:\mcL\to\R^q$ (wir nehmen direkt $d = q$ an), welche \enquote{testet}, wie gut dieses Ergebnis den gesetzten Rahmenbedingungen entspricht: Liefert sie den Wert $0$, so ist die Eingabelösung exakt gewesen. Wir definieren also als Bedingungsfunktion
        \[
            B:=\fdef{u''(t) + \frac{2\cdot m}{\hbar^2}\cdot(S - V(t))\cdot u(t)}{(u,t,S)\in C^2(\R)\times\R\times\R^2}.
        \]
        Das verwendete Potential $V$ ist dabei definiert durch $V := \bigl(\mbbEins_{\R_{<0}}(x)\cdot\infty + \mbbEins_{\R_{\geq 0}}(x)\cdot x\bigr)_{x\in\R}$. Damit haben wir eine Funktion $T$ der Form 
        \[
            T:=\fdef{\bigl(B\circ (A^{-1}_{t_0,F}(S),t_0,S)\bigr)(S,(t,t_0),F)}_{(S,(t,t_0),F)\in\R^2\times\R^2\times C^0(\I,\R)}.
        \]
            

\end{document}
