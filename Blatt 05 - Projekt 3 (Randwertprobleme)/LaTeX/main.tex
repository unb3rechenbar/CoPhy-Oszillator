\documentclass[
    oneside,
    ngerman,
    footinclude=false,
    captions=tableheading,
    DIV=12
]{scrartcl}


\usepackage{UniLaTeXPackage}

\ihead{Tom Folgmann,\\David Jannack}
\chead{Daniel Kazenwadel\\Blatt 05}
\ohead{\today}

\begin{document}
    \aufgabe{}
        \subaufgabe{}
            Zur Vorbereitung der Implementierung betrachten wir die gegebene \emph{Poisson Differentialgleichung} der Form $d^2\Phi = \mcf$ für $\Phi\in C^2(\R^3,\R)$ und $\mcf\in C^0(\R^3,\R)$. Für die gegebene Ladungsverteilung $\rho:=\big(\exp(-r)/(8\cdot\pi)\big)_{r\in\R_{>0}}$ können wir die DGL unter Drehsymmetrie reduzieren auf $d^2\varphi = f$, für angepasstes $\varphi\in C^2(\R)$ und $f\in C^0(\R)$. Mit der Identifikation $\varphi = g\circ \rho$ können wir dann die DGL aufstellen durch 
            \[
                d^2(g\circ \rho) = f.
            \] 

        \subaufgabe{}

    \aufgabe{}
        \subaufgabe{}
            Wir stellen zunächst die Schrödingergleichung $Hf = \lambda\cdot f$ für $f\in C^2(\R)$ in die nötige Form $d^2f = \fdef{F(t,f(t))}_{t\in D_f}$ um. Wir finden 
            \begin{align*}
                \nbra{-\frac{\hbar^2}{2m}\cdot d^2}(f) + f\cdot V = \lambda\cdot f\iff d^2f = \frac{2m}{\hbar^2}\cdot\big(\lambda - V\big)\cdot f.
            \end{align*}
            Wir definieren also die durch $\N$ abgezählte rechte Seite als
            \[
                n\mapsto F_n:=\fdef{\frac{2m}{\hbar^2}\cdot\big(\lambda_n - V(x)\big)\cdot f_n(x)}{t\in D_f},\quad V:=\fdef{\begin{cases}
                    x & x\geq 0\\
                    \infty & \text{sonst}
                \end{cases}}{x\in D_f}.
            \]
            Wir setzen $D_f:=\R$ und implementieren die rechte Seite.
            

\end{document}
