\documentclass{article}

\usepackage{kern}

\geometry{
    a4paper, 
    twoside=true, 
    bindingoffset=1.0471975512cm, 
    left=3.5cm, 
    right=3.5cm, 
    top=3.5cm, 
    bottom=3.5cm
}

\chead{\textit{Tom Folgmann}}

%% ---- Blattauswahl ---- %%

\ofoot{Blatt i}
\def\Datum{xx.yy.2023}

%% ---- Kursauswahl ---- %%

% \ihead{\textbf{Funktionentheorie}\\\Datum}                                % Funktionentheorie
% \ihead{\textbf{Integrierter Kurs 4:\\Experimentalphysik II}\\\Datum}      % IK4 Experimentalphysik
% \ihead{\textbf{Integrierter Kurs 4:\\Theoretische Physik II}\\\Datum}     % IK4 Theoretische Physik
% \ihead{\textbf{Mathematische Grundlagen\\der Quantenmechanik}\\\Datum}    % Mathematische Grundlagen der Quantenmechanik
% \ihead{\textbf{Numerik gewöhnlicher\\Differentialgleichungen}\\\Datum}    % Numerik gewöhnlicher Differentialgleichungen
% \ihead{\textbf{Graphical Methods:\\Games,Visualiszation,\\Film}\\\Datum}  % Computergraphics
\ihead{\textbf{Computerphysik I}\\\Datum}                                 % Computerphysik

%% ---- Tutorauswahl ---- %%

\cfoot{Tutor: \textit{Paul}}            % Funktionentheorie
% \cfoot{Tutor: \textit{Tobias}}          % IK4 Experimentalphysik
% \cfoot{Tutor: \textit{Wolf-Rüdiger}}    % IK4 Theoretische Physik
% \cfoot{Tutor: \textit{Patrick}}         % Mathematische Grundlagen der Quantenmechanik
% \cfoot{Tutor: \textit{Michael\\Junk}}   % Numerik gewöhnlicher Differentialgleichungen
% \cfoot{Tutor: \textit{Patrick}}         % Computergraphics
\cfoot{Tutor: \textit{??}}              % Computerphysik




\begin{document}
    \aufgabe{}
        \subaufgabe{}
            Sei die Kraft auf ein Teilchen durch die Funktion
            \[F_s:=\fdef{\sum_{i\in[s]}\frac{1}{\dabs{r - R_i}{2}}\cdot (r-R_i)}{r\in\R^3}\]
            gegeben, wobei $R$ das \emph{Magnetorttupel} mit Vektoren $R_i\in\R^3$ ist. Die Bewegungsgleichung ist dann gegeben durch $m\cdot r''(t) = (F_s\circ r)(t)$. Für $s = 3$ erhalten wir in der Auswertung für einen Weg $r:\R\to\R^3$ die rechte Seite $F$ mit der Definition
            \[f:=\fdef{\frac{1}{m}\cdot\sum_{i\in[3]}\frac{x-R_i}{\dabs{x-R_i}{2}}}{(t,x)\in\R\times\R^3}.\]
            Die Reduktion der Ordnung ergibt mit
            \[v(t):=\begin{pmatrix}
                x(t)\\x'(t)
            \end{pmatrix},\quad v'(t)=\begin{pmatrix}
                v_2^*(t)\\f(t,v_1^*(t))
            \end{pmatrix}.\]
            Damit ergibt sich die vektorwertige rechte Seite
            \[F:=\fdef{\begin{pmatrix}
                x_2\\f(t,x_1)
            \end{pmatrix}}{(t,x)\in\R\times\R^3}\implies F(t,x(t)) = \begin{pmatrix}
                x'(t)\\\frac{1}{m}\cdot\sum_{i\in[3]}\frac{x(t)-R_i}{\dabs{x(t)-R_i}{2}}
            \end{pmatrix}\]
            für eine Lösung $x:\R\to(\R^2)^2$. Im Reibungs- und Federfall erweitere $f$ derart, daß beide berücksichtigt werden:
            \[\tilde f:=\fdef{f(t,x_1) - \gamma\cdot x_2 - k\cdot x_1}{(t,x)\in\R\times(\R^2)^2}.\]
            Damit ergibt sich die neue rechte Seite
            \[\tilde F:=\fdef{\begin{pmatrix}
                x_1\\\tilde f(t,x_1)
            \end{pmatrix}}{(t,x)\in\R\times(\R^2)^2}.\]

        \subaufgabe{}
        
        \subaufgabe{}

        \subaufgabe{}

        \subaufgabe{}

        \subaufgabe{}
        
\end{document}