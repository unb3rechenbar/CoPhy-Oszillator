\documentclass[
    oneside,
    ngerman,
    footinclude=false,
    captions=tableheading,
    DIV=12
]{scrartcl}


\usepackage{UniLaTeXPackage}

\ihead{Tom Folgmann,\\David Jannack}
\chead{TUTOR\\Blattsammlung}
\ohead{\today}

\begin{document}
    \aufgabe{}
        \subaufgabe{}
            \paragraph*{Lösung mit Euler (explizit)}
                Die folgenden Graphen zeigen je die zeitliche Entwicklung von Ort, Geschwindigkeit und Energie, sowie die Ort-Geschdingikeits-Phasenräume für verschiedene Zeitschritte.     

                \subparagraph*{Zeitschritt $0.1$}\,
                \begin{figure}[H]
                    \centering
                    \begin{subfigure}[b]{0.45\textwidth}
                        \centering
                        \includegraphics[width=\textwidth]{Bilddateien/expEulerA1(a)-01-0-x.png}
                        \caption{Orts-Zeit-Diagramm}
                        \label{fig:expEulerA1(a)-01-0-x}
                    \end{subfigure}
                    \hfill
                    \begin{subfigure}[b]{0.45\textwidth}
                        \centering
                        \includegraphics[width=\textwidth]{Bilddateien/expEulerA1(a)-01-0-v.png}
                        \caption{Geschdingikeits-Zeit-Diagramm}
                        \label{fig:expEulerA1(a)-01-0-v}
                    \end{subfigure}
               \end{figure}
               Beide Diagramme zeigen eine harmonische Schwingung mit gut konstanter Amplitude, wobei beide Schwingungen phasenversetzt sind. Wir haben also eine hohe Übereinstimmung mit der Theorie. Im unteren Diagramm ist die Energie und der Phasenraum für die Schwingung gezeigt.
               
               \begin{figure}[H]
                \centering
                \begin{subfigure}[b]{0.45\textwidth}
                    \centering
                    \includegraphics[width=\textwidth]{Bilddateien/expEulerA1(a)-01-E.png}
                    \caption{Energie-Zeit-Diagramm}
                    \label{fig:expEulerA1(a)-01-0-E}
                \end{subfigure}
                \hfill
                \begin{subfigure}[b]{0.45\textwidth}
                    \centering
                    \includegraphics[width=\textwidth]{Bilddateien/expEulerA1(a)-01-0-xv.png}
                    \caption{Ort-Geschdingikeits-Phasenraum}
                    \label{fig:expEulerA1(a)-01-0-xv}
                \end{subfigure}
                \end{figure}
                Die Energie führt bei diesem Zeitschritt noch größere Schwingungen aus, bleibt aber im Mittel konstant. Dies man auch gut daran ablesen kann, dass im Phasenraum eine zwar geschlossene, aber unsymmetrische Ellipse zu sehen ist.

                \subparagraph*{Zeitschritt $0.01$}\,
                \begin{figure}[H]
                    \centering
                    \begin{subfigure}[b]{0.45\textwidth}
                        \centering
                        \includegraphics[width=\textwidth]{Bilddateien/expEulerA1(a)-001h-x.png}
                        \caption{Orts-Zeit-Diagramm}
                        \label{fig:expEulerA1(a)-001-0-x}
                    \end{subfigure}
                    \hfill
                    \begin{subfigure}[b]{0.45\textwidth}
                        \centering
                        \includegraphics[width=\textwidth]{Bilddateien/expEulerA1(a)-001h-v.png}
                        \caption{Geschdingikeits-Zeit-Diagramm}
                        \label{fig:expEulerA1(a)-001-0-v}
                    \end{subfigure}
                \end{figure}
                
                \begin{figure}[H]
                \centering
                \begin{subfigure}[b]{0.45\textwidth}
                    \centering
                    \includegraphics[width=\textwidth]{Bilddateien/expEulerA1(a)-001h-E.png}
                    \caption{Energie-Zeit-Diagramm}
                    \label{fig:expEulerA1(a)-001-0-E}
                \end{subfigure}
                \hfill
                \begin{subfigure}[b]{0.45\textwidth}
                    \centering
                    \includegraphics[width=\textwidth]{Bilddateien/expEulerA1(a)-001-0-xv.png}
                    \caption{Ort-Geschdingikeits-Phasenraum}
                    \label{fig:expEulerA1(a)-001-0-xv}
                \end{subfigure}
                \end{figure}
                Auch hier schwingt die Energie, aber mit kleinerer Ampltiude.

                \subparagraph*{Zeitschritt $0.001$}\,
                \begin{figure}[H]
                    \centering
                    \begin{subfigure}[b]{0.45\textwidth}
                        \centering
                        \includegraphics[width=\textwidth]{Bilddateien/expEulerA1(a)-0001-0-x.png}
                        \caption{Orts-Zeit-Diagramm}
                        \label{fig:expEulerA1(a)-0001-0-x}
                    \end{subfigure}
                    \hfill
                    \begin{subfigure}[b]{0.45\textwidth}
                        \centering
                        \includegraphics[width=\textwidth]{Bilddateien/expEulerA1(a)-0001-0-v.png}
                        \caption{Geschdingikeits-Zeit-Diagramm}
                        \label{fig:expEulerA1(a)-0001-0-v}
                    \end{subfigure}
                \end{figure}
                
                \begin{figure}[H]
                \centering
                \begin{subfigure}[b]{0.45\textwidth}
                    \centering
                    \includegraphics[width=\textwidth]{Bilddateien/expEulerA1(a)-0001-E.png}
                    \caption{Orts-Zeit-Diagramm}
                    \label{fig:expEulerA1(a)-0001-0-E}
                \end{subfigure}
                \hfill
                \begin{subfigure}[b]{0.45\textwidth}
                    \centering
                    \includegraphics[width=\textwidth]{Bilddateien/expEulerA1(a)-0001-0-xv.png}
                    \caption{Energie-Geschdingikeits-Phasenraum}
                    \label{fig:expEulerA1(a)-0001-0-xv}
                \end{subfigure}
            \end{figure}
            Die Energie variiert hier nur noch sehr wenig, was sich in einer recht idealen Ellipse im Phasenraum äußert.

        \paragraph{Lösung mit Euler (implizit)}
            \subparagraph*{Zeitschritt $0.01$}\,
            \begin{figure}[H]
                \centering
                \begin{subfigure}[b]{0.45\textwidth}
                    \centering
                    \includegraphics[width=\textwidth]{Bilddateien/impEulerA1(a)-001h-x.png}
                    \caption{Orts-Zeit-Diagramm}
                    \label{fig:impEulerA1(a)-001-0-x}
                \end{subfigure}
                \hfill
                \begin{subfigure}[b]{0.45\textwidth}
                    \centering
                    \includegraphics[width=\textwidth]{Bilddateien/impEulerA1(a)-001h-v.png}
                    \caption{Geschdingikeits-Zeit-Diagramm}
                    \label{fig:impEulerA1(a)-001-0-v}
                \end{subfigure}
            \end{figure}
            
            \begin{figure}[H]
            \centering
            \begin{subfigure}[b]{0.45\textwidth}
                \centering
                \includegraphics[width=\textwidth]{Bilddateien/impEulerA1(a)-001h-E.png}
                \caption{Orts-Zeit-Diagramm}
                \label{fig:impEulerA1(a)-0001-0-E}
            \end{subfigure}
            \end{figure}
            Das implizite Eulerverfahren zeigt beim gleichen Zeitrschritt minimal kleinere Schwankungen als das explizite.

        \paragraph*{Lösung mit leap-frog}
            \subparagraph*{Zeitschritt $0.1$}\,
            \begin{figure}[H]
                \centering
                \begin{subfigure}[b]{0.45\textwidth}
                    \centering
                    \includegraphics[width=\textwidth]{Bilddateien/LLA1(a)-01-0-x.png}
                    \caption{Orts-Zeit-Diagramm}
                    \label{fig:LLA1(a)-01-0-x}
                \end{subfigure}
                \hfill
                \begin{subfigure}[b]{0.45\textwidth}
                    \centering
                    \includegraphics[width=\textwidth]{Bilddateien/LLA1(a)-01-0-v.png}
                    \caption{Geschdingikeits-Zeit-Diagramm}
                    \label{fig:LLA1(a)-01-0-v}
                \end{subfigure}
            \end{figure}
            In beiden Diagrammen lässt sich eine Dämpfung der Schwingung erkennen, obwohl ein Dämpfungskoeffizient von $\gamma=0$ vorliegt. Für diesen Zeitrschritt ist das leap-frog-Verfahren also noch dem Eulerverfahren unterlegen.
        
            \begin{figure}[H]
            \centering
            \begin{subfigure}[b]{0.45\textwidth}
                \centering
                \includegraphics[width=\textwidth]{Bilddateien/LLA1(a)-01-E.png}
                \caption{Energie-Zeit-Diagramm}
                \label{fig:LLA1(a)-01-0-E}
            \end{subfigure}
            \hfill
            \begin{subfigure}[b]{0.45\textwidth}
                \centering
                \includegraphics[width=\textwidth]{Bilddateien/LLA1(a)-01-0-xv.png}
                \caption{Ort-Geschdingikeits-Phasenraum}
                \label{fig:LLA1(a)-01-0-xv}
            \end{subfigure}
            \end{figure}
            Wie aufgrund der Dämpfung des Schwingungen zu erwarten ist, fällt die Enerige mit der Zeit (hat aber periodische kleine lokale Maxima). Im Phasendiagramm äußert sich das in Form einer Spirale.

            \subparagraph*{Zeitschritt $0.01$}\,
            \begin{figure}[H]
                \centering
                \begin{subfigure}[b]{0.45\textwidth}
                    \centering
                    \includegraphics[width=\textwidth]{Bilddateien/LLA1(a)-001-0-x.png}
                    \caption{Orts-Zeit-Diagramm}
                    \label{fig:LLA1(a)-001-0-x}
                \end{subfigure}
                \hfill
                \begin{subfigure}[b]{0.45\textwidth}
                    \centering
                    \includegraphics[width=\textwidth]{Bilddateien/LLA1(a)-001-0-v.png}
                    \caption{Geschdingikeits-Zeit-Diagramm}
                    \label{fig:LLA1(a)-001-0-v}
                \end{subfigure}
            \end{figure}
            Mit kleineren Zeitschritte ist auch die Dämpfung der Schwingung kleiner, es handelt sich also tatsächlich um einen numerischen Fehler.
            
            \begin{figure}[H]
            \centering
            \begin{subfigure}[b]{0.45\textwidth}
                \centering
                \includegraphics[width=\textwidth]{Bilddateien/LLA1(a)-001-E.png}
                \caption{Energie-Zeit-Diagramm}
                \label{fig:LLA1(a)-001-0-E}
            \end{subfigure}
            \hfill
            \begin{subfigure}[b]{0.45\textwidth}
                \centering
                \includegraphics[width=\textwidth]{Bilddateien/LLA1(a)-001-0-xv.png}
                \caption{Ort-Geschdingikeits-Phasenraum}
                \label{fig:LLA1(a)-001-0-xv}
            \end{subfigure}
            \end{figure}
            Im Phasendiagramm äußert sich die kleinere Energiedissizipation dadurch, dass die Spirale sehr viel länger braucht, um abzufallen.

            \subparagraph*{Zeitschritt $0.001$}\,
            \begin{figure}[H]
                \centering
                \begin{subfigure}[b]{0.45\textwidth}
                    \centering
                    \includegraphics[width=\textwidth]{Bilddateien/LLA1(a)-0001-0-x.png}
                    \caption{Orts-Zeit-Diagramm}
                    \label{fig:LLA1(a)-0001-0-x}
                \end{subfigure}
                \hfill
                \begin{subfigure}[b]{0.45\textwidth}
                    \centering
                    \includegraphics[width=\textwidth]{Bilddateien/LLA1(a)-0001-0-v.png}
                    \caption{Geschdingikeits-Zeit-Diagramm}
                    \label{fig:LLA1(a)-0001-0-v}
                \end{subfigure}
            \end{figure}
            Für $h=0.001$ ist die Amplitude nun nahezu konstant.
            
            \begin{figure}[H]
            \centering
            \begin{subfigure}[b]{0.45\textwidth}
                \centering
                \includegraphics[width=\textwidth]{Bilddateien/LLA1(a)-0001-E.png}
                \caption{Energie-Zeit-Diagramm}
                \label{fig:LLA1(a)-0001-0-E}
            \end{subfigure}
            \hfill
            \begin{subfigure}[b]{0.45\textwidth}
                \centering
                \includegraphics[width=\textwidth]{Bilddateien/LLA1(a)-0001-0-xv.png}
                \caption{Ort-Geschdingikeits-Phasenraum}
                \label{fig:LLA1(a)-0001-0-xv}
            \end{subfigure}
            \end{figure}
            Auch die Energie hat eine sehr viel kleinere Steigung und die Spirale ähnelt sehr einer Ellipse. Allerdings deutet die im Vergleich zum expliziten Eulerverfahren dickeren "Ränder" an, dass dieses immernoch akkurater ist hier.


        \subaufgabe{}

        \subaufgabe{}

        \subaufgabe{}
            

    \newpage
    \subsection*{Hauptcode}
        \lstinputlisting[language=C++]{../src/main.cpp}

    \subsection*{Headerdateien}
        \subsubsection*{Einschrittverfahren}
            \lstinputlisting[language=C++]{../src/header/Einschritt.hpp}

        \subsubsection*{Mehrschrittverfahren}
            \lstinputlisting[language=C++]{../src/header/Mehrschritt.hpp}

        \subsubsection*{Treibende Kräfte}
            \lstinputlisting[language=C++]{../src/header/Funktionsspielereien.hpp}

\end{document}
